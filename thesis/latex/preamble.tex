%Dokumentklasse
\documentclass[a4paper,12pt,makeidx,twoside,openright,numbers=noenddot]{scrreprt}
\usepackage[left=3cm, right=3cm, bottom=3cm, top=3cm]{geometry}		% Formatierung der Seitenränder
%\usepackage[left=3cm, right=3cm, bottom=3cm, top=3cm]{geometry}		    % Formatierung der Seitenränder
%\usepackage[onehalfspacing]{setspace}							        % verwendet für größeren Zeilenabstand

% ============= Packages =============

% Dokumentinformationen
\usepackage[
	pdftitle={Self-sovereign Identity: Development of an Implementation-based Evaluation Framework for Verifiable Credential SDKs},
	pdfsubject={},
	pdfauthor={Philipp Bolte},
	pdfkeywords={ssi, sdk, verifiable, credentials, evaluation},
	%Links nicht einrahmen
	hidelinks
]{hyperref}

\usepackage[utf8]{inputenc}
\usepackage[english]{babel}
\usepackage[T1]{fontenc}
\usepackage{units}
\usepackage{pdfpages}
\usepackage{listings}
\usepackage{svg}					% Zur Einbindung von scalable vector graphics
\usepackage{subcaption}				% Zum Einbinden von Untergrafiken
\usepackage[gen]{eurosym}			% Zur Verwendung des Eurozeichens
\usepackage{amssymb}
\usepackage{graphicx}
\graphicspath{{img/}}				
\usepackage{fancyhdr}				% Style Package für's Seitenlayout
\usepackage{xcolor}					% anpassen von Farben
\usepackage{microtype} 				% schönerer Blocksatz!
\usepackage{calc}  					
\usepackage{enumitem}				% zb für align der description
\usepackage[font=small,labelfont=bf]{caption}
\usepackage[printonlyused]{acronym}	% für Abkürzungen
\usepackage{multicol}
\usepackage{booktabs}
\usepackage{textcomp}
\usepackage{listings}				% Einbindung von Code in LateX
\usepackage{setspace}
\usepackage{threeparttable} 		% für Fußnoten innerhalb einer Tabelle
\usepackage[export]{adjustbox}
\usepackage{csquotes}
\usepackage{bera}% optional: just to have a nice mono-spaced font
\usepackage{listings}
\usepackage{color}
\usepackage{pifont}
\usepackage{makecell}
\usepackage{vcell}
\usepackage{minted}
\usepackage{rotating}
\usepackage{xurl}
\usepackage{longtable}

% verschiedene Schriftarten
%\usepackage{times} 				% times font
%\usepackage{palatino}			 	% Palatino font
\usepackage{lmodern}				% Lmodern sans und serif
%\usepackage{libertine}				% Linux Libertine und Biolinum

%\usepackage{fontspec}				% Nutzen in Kombination mit LuaLaTeX, um Systemschriften einzubinden
%\setmainfont{Georgia}				% beispielsweise Georgia, aber auch jede andere Schrift, die auf dem PC vorhanden ist


% zusätzliche Schriftzeichen der American Mathematical Society
\usepackage{amsfonts}
\usepackage{amsmath}

\usepackage[numbers, comma]{natbib}		% Einstellung des Zitierstils
\bibliographystyle{alphaurl}			% Angepasster Stil für deutsche Sprache

\setcounter{secnumdepth}{3}				% Nummerierungsebene anpassen -> 3 = subsubsection werden nummeriert
\setcounter{tocdepth}{3}   				% gliederungsebenen im Inhaltsverzeichnis -> erstmal nur zur übersicht was nicht vergessen werden darf

\definecolor{deepblue}{rgb}{0,0,0.5}
\definecolor{deepred}{rgb}{0.6,0,0}
\definecolor{deepgreen}{rgb}{0,0.5,0}

\DeclareUnicodeCharacter{2212}{-}


% ============= Kopf- und Fußzeile =============
\pagestyle{fancy}

%% Formatierung der Kopf- und Fußzeile
\fancyhead{}
\fancyhead[RO,LE]{\thepage}
\fancyhead[RE]{\leftmark}
\fancyhead[LO]{\rightmark}
%%
\fancyfoot{}

\renewcommand{\headrulewidth}{0.4pt}		% Bei zweiseitigem Dokument ausschließlich Linie in Kopfzeile
\renewcommand{\chaptermark}[1]{\markboth{\thechapter\ #1}{}}
\renewcommand{\sectionmark}[1]{\markright{\thesection\ #1}}

% ============= Package Einstellungen & Sonstiges ============= 
% Besondere Trennungen
\hyphenation{Um-ge-bungs-tem-pe-ra-tur Um-ge-bungs-tem-pe-ra-tur-en Rauch-gas-tem-pe-ra-tur Aus-tritts-tem-pe-ra-tur}

% Einstellung wie Code innerhalb der Arbeit gesetzt werden soll:
\lstdefinestyle{Style}{
	columns=flexible,
	basicstyle=\ttfamily}
\lstset{ 
	backgroundcolor=\color{white},   % choose the background color; you must add \usepackage{color} or 
	 % should come as last argument
	basicstyle=\footnotesize,        % the size of the fonts that are used for the code
	breakatwhitespace=false,         % sets if automatic breaks should only happen at whitespace
	breaklines=true,                 % sets automatic line breaking
	captionpos=none,                 % sets the caption-position to bottom
	commentstyle=\color{deepblue},   % comment style
	deletekeywords={...},            % if you want to delete keywords from the given language
	escapeinside={\%*}{*)},          % if you want to add LaTeX within your code
	extendedchars=true,              % lets you use non-ASCII characters; for 8-bits encodings only, does not work with UTF-8
	firstnumber=1,               	 % start line enumeration with line 1000
	frame=single,	                 % adds a frame around the code
	keepspaces=true,                 % keeps spaces in text, useful for keeping indentation of code (possibly needs columns=flexible)
	keywordstyle=\color{blue},       % keyword style
	language=Python,                 % the language of the code
	morekeywords={*,...},            % if you want to add more keywords to the set
	numbers=left,                    % where to put the line-numbers; possible values are (none, left, right)
	numbersep=5pt,                   % how far the line-numbers are from the code
	emphstyle=\color{deepred},
	%numberstyle=\tiny\color{mygray}, % the style that is used for the line-numbers
	rulecolor=\color{black},         % if not set, the frame-color may be changed on line-breaks within not-black text (e.g. comments (green here))
	showspaces=false,                % show spaces everywhere adding particular underscores; it overrides 'showstringspaces'
	showstringspaces=false,          % underline spaces within strings only
	showtabs=false,                  % show tabs within strings adding particular underscores
	stepnumber=2,                    % the step between two line-numbers. If it's 1, each line will be numbered
	stringstyle=\color{deepgreen},     % string literal style
	tabsize=2,	                   	 % sets default tabsize to 2 spaces
	title=\lstname                   % show the filename of files included with \lstinputlisting; also try caption instead of title
}

\definecolor{eclipseStrings}{RGB}{42,0.0,255}
\definecolor{eclipseKeywords}{RGB}{127,0,85}
\colorlet{numb}{magenta!60!black}
\lstdefinelanguage{json}{
    basicstyle=\fontfamily{fvm}\selectfont\linespread{0.8},
    commentstyle=\color{eclipseStrings}, % style of comment
    stringstyle=\color{eclipseKeywords}, % style of strings
    numbers=left,
    numberstyle=\scriptsize,
    stepnumber=1,
    numbersep=8pt,
    showstringspaces=false,
    breaklines=true,
    frame=lines,
    backgroundcolor=\color{white}, %only if you like
    string=[s]{"}{"},
    comment=[l]{:\ "},
    morecomment=[l]{:"},
    literate=
        *{0}{{{\color{numb}0}}}{1}
         {1}{{{\color{numb}1}}}{1}
         {2}{{{\color{numb}2}}}{1}
         {3}{{{\color{numb}3}}}{1}
         {4}{{{\color{numb}4}}}{1}
         {5}{{{\color{numb}5}}}{1}
         {6}{{{\color{numb}6}}}{1}
         {7}{{{\color{numb}7}}}{1}
         {8}{{{\color{numb}8}}}{1}
         {9}{{{\color{numb}9}}}{1}
}

\lstdefinelanguage{JavaScript}{
  morekeywords=[1]{break, continue, delete, else, for, function, if, in,
    new, return, this, typeof, var, void, while, with},
  % Literals, primitive types, and reference types.
  morekeywords=[2]{false, null, true, boolean, number, undefined,
    Array, Boolean, Date, Math, Number, String, Object},
  % Built-ins.
  morekeywords=[3]{eval, parseInt, parseFloat, escape, unescape},
  sensitive,
  morecomment=[s]{/*}{*/},
  morecomment=[l]//,
  morecomment=[s]{/**}{*/}, % JavaDoc style comments
  morestring=[b]',
  morestring=[b]"
}[keywords, comments, strings]

\lstalias[]{ES6}[ECMAScript2015]{JavaScript}
\lstdefinelanguage[ECMAScript2015]{JavaScript}[]{JavaScript}{
  morekeywords=[1]{await, async, case, catch, class, const, default, do,
    enum, export, extends, finally, from, implements, import, instanceof,
    let, static, super, switch, throw, try, interface, public, private},
  morestring=[b]` % Interpolation strings.
}

% Requires package: color.
\definecolor{mediumgray}{rgb}{0.3, 0.4, 0.4}
\definecolor{mediumblue}{rgb}{0.0, 0.0, 0.8}
\definecolor{forestgreen}{rgb}{0.13, 0.55, 0.13}
\definecolor{darkviolet}{rgb}{0.58, 0.0, 0.83}
\definecolor{royalblue}{rgb}{0.25, 0.41, 0.88}
\definecolor{crimson}{rgb}{0.86, 0.8, 0.24}

\lstdefinestyle{JSES6Base}{
  backgroundcolor=\color{white},
  basicstyle=\fontsize{11}{13}\fontfamily{fvm}\selectfont\linespread{0.8},
  breakatwhitespace=false,
  breaklines=false,
  captionpos=b,
  columns=fullflexible,
  commentstyle=\color{mediumgray}\upshape,
  emph={},
  emphstyle=\color{crimson},
  extendedchars=true,  % requires inputenc
  fontadjust=true,
  frame=lines,
  identifierstyle=\color{black},
  keepspaces=true,
  keywordstyle=\color{mediumblue},
  keywordstyle={[2]\color{darkviolet}},
  keywordstyle={[3]\color{royalblue}},
  numbers=left,
  numbersep=8pt,
  numberstyle=\scriptsize,%\tiny\color{black},
  rulecolor=\color{black},
  showlines=true,
  showspaces=false,
  showstringspaces=false,
  showtabs=false,
  stringstyle=\color{forestgreen},
  stepnumber=1,
  title=\lstname,
  upquote=true  % requires textcomp
}

\lstdefinestyle{JavaScript}{
  language=JavaScript,
  style=JSES6Base
}
\lstdefinestyle{ES6}{
  language=ES6,
  style=JSES6Base
}

% nicht einrücken nach Absatz
\setlength{\parindent}{0pt}
\usepackage{parskip}		 			% verhindert einrücken und setzt einen kleinen Absatz

\renewcommand{\arraystretch}{1.2}		% Abstand innerhalb der Tabellen einstellen