% work in progress

\chapter{Evaluation Framework}\label{chapter: framework}
    %Allgemeiner Einstieg; Warum noch mal eval frame? Ziel.
    
    In this chapter, a new developer-oriented evaluation framework for \ac{SSI} solutions based on all previous findings is presented. First, requirements for the framework are defined to clarify which areas the framework should cover. Then, the framework is built up step by step with its indexes, criteria and questions, and finally applied to the four solutions of the reference implementation in the end. 
    
	\section{Requirements}
	%Grundlegende Anforderungen -> developer focus, was soll abgebildet werden (siehe implementierung, umfrage)
	
	Unlike previous evaluations from the literature, this framework is not intended to address architectures, governance models, or cover any sets of ideational principles. The goal is to provide a practical tool for developers to pragmatically evaluate, depending on the use case, \ac{SSI} solutions for their suitability. This should accelerate the selection and evaluation process and reduce the hurdles for integrating \ac{SSI} technologies into projects. To meet these objectives, the following requirements are initially defined:
	
	\begin{itemize}
	    \item \textit{Developer-oriented}: A practical value should be created by mapping various facets and requirements of a developer. On the one hand, this can involve the functionalities, but also the toolset and its documentation. For this purpose, experiences from the implementation of the reference implementation as well as requirements from the expert survey are used. This is to ensure that general but also domain-specific points are included. % practical, get things done (functionality, ...)
	    \item \textit{Expert-oriented}: As mentioned in the first point, requirements and opinions from experts in the field are to be incorporated into the framework in addition to insights from this work. This is to ensure that a broad field is covered, but also that requirements that are actually relevant to the domain are part of the framework. The findings from chapter \ref{chapter: expert} are used for this purpose.  % Know-how & experience from field
	    \item \textit{Technologies}: Since \ac{SSI} is still a relatively young domain, many solutions and standards are either very young, not ready or not even defined yet. Therefore, it is necessary that the coverage of some important standards is also represented in the framework. Thereby, it can be recognized whether corresponding solutions provide the most current and established technologies. In addition, it must be taken into account whether governance models behind the individual solutions can react to the fast-moving area with appropriate measures to e.g. add support for new technologies.  % tech stack, Technologie support, future? (governance)
	    \item \textit{Unopinionated}: Similar to the reference implementation, the evaluation framework should be use case agnostic. This would also imply general weightings of different areas of the framework for a heterogeneous set of use cases. Nevertheless, a generally valid evaluation with regard to different requirements would not be meaningful or even possible. Therefore, any weighting should be dispensed  at this point, and it should be left to the developer to decide which areas have a higher or lower priority.  % Weighing 
	\end{itemize}
	
	Thus, the foundations for the development of the framework itself have been laid at this point. In the next section, the actual framework is developed and presented, taking the requirements into account.
	
	\section{Framework}
	%Prozess: Grundlegende Bereiche/ Kategorien definieren (als Frage), definieren der Kriterien -> Fragen, Darstellung in Tabellenform, Punktevergabesystem
	
    As mentioned previously, the development of the framework is based on the experience gained from the development of the reference implementation as well as on the results of the expert survey. From the latter, first indexes were established in section \ref{section: expert results}, which will be presented and extended in this section. The corresponding five indices with their criteria are described below.
    
    \begin{enumerate}
        \item \textit{Functionality}: The first index is focused on the question, which features the solution offers the developer, in order to solve its \ac{SSI} related tasks. This includes flow coverage, i.e. the extent to which the \ac{vc} lifecycle can be mapped by the solution, but also the support of standards. These are particularly important in an area where openness and interoperability between diverse user-managed flows and applications are intrinsic values. Both criteria were relevant both in the expert questionnaire and in the development of the reference implementation.
        \item \textit{Flexibility}: This is mainly about how freely a developer can interact with the solution. This was a rather small area in the survey, but was much more apparent in the implementation in chapter \ref{chapter: implementation}. On the one hand, this contains extensibility, thus to what extent the functionality of the solution can be extended by own efforts and whether these can be integrated also by e.g. pull requests into the public code base. Furthermore, it includes the complete deployment field, i.e. on which type of devices this solution can be used, but also whether there are quick deployment options for the cloud, for example. The latter in particular is relevant during the development phase for setting up test environments as quickly and conveniently as possible. Finally, Flexibility also includes the platform criterion, which looks at the ways in which the solution can be addressed. This includes, for example, the number of supported programming languages and the existence of REST APIs to enable a platform-independent use of the solutions.
        \item \textit{Operability}: This index primarily considers how usable the solution is in terms of the developer experience. This includes obvious criteria such as support, documentation and, for example, the maturity of the solution. It also includes attention to industry standards, creating a common conversational and well-thought-out technological foundation. The last criterion defined was overhead, which considers the initial effort required to get the solution up and running. The contents of the index were often part of the expert survey and also appeared in the reference implementation.
        \item \textit{Dependency}: This index was not directly part of the experts' answers, but was partly addressed secondary, for example by Stefan Adolf, and should not be disregarded considering the principles of \ac{SSI}. This involves questions about the extent to which a developer is dependent on the solution creators. For example, this includes the keys criterion, which is whether a developer has complete control and access to the private keys of its \acp{DID}. In addition, it considers whether the tech stack can be under full control of the developer. Finally, this also includes the cost of the solution and whether there are developer-specific test accounts, for example. Especially the latter is quite helpful for a development phase in order not to generate high costs.
        \item \textit{Involvement}: In the last index, the involvement of the solution creators in the \ac{SSI} community is considered and thus the closeness to technological changes. This includes participation in the development of standards as well as open-source libraries. Additionally, it includes the commitment related to the own product, i.e. whether the solution is being developed further, whether it uses existing community work or whether it also incentivizes community help. This index was created primarily from the results of the expert survey.
    \end{enumerate}
	
    
    \setlength\LTleft{0pt}
    \setlength\LTright{0pt}
    \begin{longtable}{@{\extracolsep{\fill}}lll@{}}
    \toprule
    \textbf{Criterion}     & \textbf{Question}                                                                                               & \textbf{Type}                                                              \\* \midrule
    \endfirsthead
    %
    \endhead
    %
    \endfoot
    %
    \endlastfoot
    %
    \textbf{Functionality} &                                                                                                                 &                                                                            \\
    \textit{Flow Coverage} & \begin{tabular}[t]{@{}l@{}}FC1: What percentage of the VC lifecycle \\ can be implemented directly?\end{tabular} & double                                                                     \\
                           & \begin{tabular}[t]{@{}l@{}}FC2: What percentage of the VC lifecycle\\ is generally supported?\end{tabular}       & double \\
                           & FC3: Can a wallet be built independently? & bool  \\
                           & \begin{tabular}[t]{@{}l@{}}FC4: What pre-built wallet options are\\ supported?\end{tabular}                      & \begin{tabular}[t]{@{}l@{}}{[}mobile,\\ cloud,\\ browser{]}\end{tabular}   \\
    \textit{Standards}     & FS1: How many DID methods are supported?                                                                         & int                                                                        \\
                           & FS2: Are JSON-LD credentials supported? & bool  \\
                           & FS3: Are linked data proofs supported? & bool \\
                           & FS4: Are BBS+ or CL signatures supported? & bool \\                        & FS5: Is RevocationList2020 supported & bool  \\
                           & FS6: Is there ODIC support? & bool \\
                           \midrule
    \textbf{Flexibility}   &                                                                                                                 &                                                                            \\
    \textit{Extensibility} & FE1: Can the functionality be extended?                                                                          & bool                                                                       \\
                           & FE2: Can one contribute to the solution?                                                                         & bool                                                                       \\
    \textit{Deployment}    & FD1: What deployment options are there?                                                                          & \begin{tabular}[t]{@{}l@{}}{[}cloud, pc,\\ mobile,\\ mixed{]}\end{tabular} \\
                           & FD2: Is there a quick deployment option?                                                                         & bool                                                                       \\
    \textit{Platform}      & \begin{tabular}[t]{@{}l@{}}FP1: Is there a REST API exposing all\\ necessary functionality?\end{tabular}         & bool                                                                       \\
                           & \begin{tabular}[t]{@{}l@{}}FP2: Are multiple programming languages\\ supported by the solution?\end{tabular}            & bool \\
                           \midrule
    \textbf{Operability}   &                                                                                                                 &                                                                            \\
    \textit{Support}       & OS1: What support options are there?                                                                                 & \begin{tabular}[t]{@{}l@{}}{[}tel, mail\\ chat, forum{]}\end{tabular}      \\
    
    \textit{Documentation} & \begin{tabular}[t]{@{}l@{}}OD1: What percentage of the flow\\ is described?\end{tabular}   
                            & double  \\
                           & OD2: Is there an API Documentation? & bool\\
                           & OD3: Are there any code examples? & bool\\
    \textit{Maturity}      & OM1: Is there a release version? & bool \\  
                           & OM2: Are any features subject to change? & bool\\
                           & OM3: Is the solution older than a year? & bool\\
                           & (OM4: How many reliability issues are there?) & int\\
    \textit{Standards}     & OT1: Are industry standards being used? & bool \\  
                           & OT2: Is interoperability proven by Plugfest? & bool\\
    \textit{Overhead}      & OO1: How many LoCs are needed for set up? & int \\  
                           & OO2: How many things have to be set up? & int\\
                           \midrule
    \textbf{Dependency}    & & \\
    \textit{Keys}          & DK1: Is full control over private keys given? & bool \\  
    \textit{Stack}         & DS1: Can one own the whole tech stack? & bool \\  
                           & \begin{tabular}[t]{@{}l@{}}DS2: Is the still stack usable after\\
                           the original developers are gone?\end{tabular} & bool \\ 
    \textit{Cost}          & DC1: Is the solution free of charge? & bool \\  
                           & DC2: Is there a free developer plan? & bool \\
                           \midrule
    \textbf{Involvement}   & & \\
    \textit{Community}     & \begin{tabular}[t]{@{}l@{}}IC1: Does the solution provider participate 
                            \\in community events?\end{tabular} & bool \\
                           & \begin{tabular}[t]{@{}l@{}}IC2: How many open-source repositories have \\they worked on?\end{tabular} & int \\ 
                           & \begin{tabular}[t]{@{}l@{}}IC3: Have they co-worked on \ac{SSI}-related \\standards?\end{tabular} & bool \\ 
    \textit{Product}       & IP1: Is the solution being worked on? & bool \\
                           & IP2: Is the solution leveraging open-source work? & bool\\
                           & \begin{tabular}[t]{@{}l@{}}IP3: Does the governance structure encourage 
                            \\community efforts?\end{tabular} & bool \\*
    \bottomrule
    \end{longtable}
    
    With regard to the evaluation and the lack of weightings, one point can be obtained with each question. For questions with more than one option, the sum of all options results in one point. To compensate for the different number of questions within the indexes, the results are normalized so that all indexes are equally weighted.
    
	\section{Results}
	%Einordnen der Lösungen in Tabelle -> Bewertung;;; Was sind die Ergebnisse? Welche Lösung eignet sich für das? Was machen welche Lösungen sehr gut, sehr schlecht? Überraschungen bezügluch Vorbetrachtung?
	
	
	
	