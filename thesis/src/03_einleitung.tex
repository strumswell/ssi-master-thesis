\chapter{Introduction}
\pagenumbering{arabic}
	The Internet has become a cornerstone of coexistence in today's world. With over 4.66 billion Internet users worldwide \cite{johnson_internet_2021}, it determines how we communicate, think, inform ourselves, and interact with one another.	As a result, huge networks of people are being created in which different cultures are coming closer together and knowledge is being shared like never before.
	A central enabler for the functioning of such interactions are digital identities, which are gaining an increasingly important role \cite{liu_blockchain-based_2020}.
	
	Over the course of our lives, we collect a large amount of digital identities from a wide variety of services, including Facebook, Twitter, WhatsApp, GitHub, LinkedIn, and many more. Because of the way we manage digital identities in the current era, users mostly own separate identities for each service or go through centralized, federated identity providers like Google or Facebook. As a result of these key developments, silos of identity data emerged, which are problematic concerning efficiency, security, and privacy. Several historical data leaks and hacks in which sensitive user data was made public show that these approaches are not suitable for managing sensitive user data. \cite{swinhoe_15_2021}. \cite{ehrlich_self-sovereign_2021}
	
	In contrast, the \ac{SSI} paradigm takes a new approach by giving back users control of their digital identities through various novel approaches. This thesis explores this new approach from a developer's point of view. In the next sections, the objectives, related work and the research approach will be discussed.
	
	\section{Objectives} % might be the wrong word; maybe objectives? Or "Motivation and Objectives"?
	
	For a successful realization of \acf{SSI} concepts, the existence of good solutions for developers is critical. This ensures that the barriers to a successful adoption of \ac{SSI} are kept to a minimum, simplifying and speeding up the entire process. A good toolset and developer experience is thus a key enabler for \ac{SSI}.
	
	With this in mind, an overview of the most important solutions\footnote{synonymous to SDKs, libraries, frameworks, and platforms} in the \ac{SSI} space will be established throughout the thesis. To scope the work accordingly, this thesis looks at the solutions in terms of how closely they can map the lifecycle of a \acf{VC}. It is intended to serve as an entry point for developers to get an overview of the capabilities of existing solutions and to give starting points for further research. 
	Furthermore, a use case agnostic reference implementation is presented that implements four of the presented solutions based on the lifecycle. It can serve developers as a basis for their own work, but above all enables practical validation and the gathering of experience during its development. In this way, the knowledge gained flows directly into a new evaluation framework, which, in addition to other software selection frameworks, can provide concrete help in selecting the most suitable solution from the developer's point of view. In addition, it can reveal shortcomings in current solutions that need to be addressed for successful adoption of \ac{SSI} in practical use cases. So the objective of this work, besides the scientific contributions, is to generate added value for the whole ecosystem.
	
	\section{Related Work}
	% What has already been done? Why is my work novel and what do I contribute to the space?
	At the current time, there does not appear to be any comparable work that addresses the topic in a manner corresponding to Section 1.1. The most similar is \cite{naik_uport_2020} who have developed a mobile wallet based on uPort that covers login, \ac{VC} issuance as well as verification. Based on the experience gained, an evaluation of uPort has been made as well.However, uPort is currently no longer being developed, and the assessment is also based on only a fraction of the VC lifecycle and basic principles for \ac{SSI}. 
	
    Another paper by \cite{kuperberg_blockchain-based_2020} defines a comprehensive evaluation framework from an enterprise perspective that, compared to other papers, also covers aspects such as user experience, technology and compliance. It is characterized by a wide range of questions that are used for the evaluation of 43 solutions. However, the list of solutions considered is outdated and missing important players (see e.g. MATTR and Trinsic). Furthermore, the assessment does not provide any practical guidance for developers. A clear analysis of the SSI-relevant features, e.g. with regard to the \ac{VC} lifecycle, does not exist.
    
    Otherwise, many papers seem to focus on theoretical foundations or evaluation of existing solution based on two things: (i) architecture \cite{gruner_relevance_2018} concerning privacy \cite{bernabe_privacy-preserving_2019}, performance \cite{bouras_distributed_2020}, use case \cite{kuperberg_blockchain-based_2020}, various variations \cite{allen_path_2016, reed_decentralized_2021, allende_lopez_self-sovereign_2020, bouras_distributed_2020, ferdous_search_2019, cameron_laws_2005} of \ac{SSI} principles \cite{van_bokkem_self-sovereign_2019, bouras_distributed_2020, dib_decentralized_2020, dunphy_first_2018, ferdous_search_2019, friedewald_self-sovereign_2020}, and (ii) the interoperability between those systems \cite{homeland_security_preventing_2020, john_dhs_2020}. This clearly shows that there is a deficit in terms of works that look at existing solutions based on their practical features and applicability from a developer's point of view. This thesis addresses some of these gaps and thus clearly contributes to the field of research.
	
	\section{Methodology}
	% Research questions, approach, methods

