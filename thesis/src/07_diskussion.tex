\chapter{Conclusion}

%Digital identities und ihre Umsetzungen sind ein Jahrzente altes Problem, was bisher nur unzureichend gelöst werden konnte. Aktuelle Ansätze führen zu noch nie da gewesenen Leaks persönlicher Daten und dem Verlust von Kontrolle über von Millionen von Menschen weltweit.

Self-sovereign Identity is a new approach towards digital identities. This thesis gives an overview over existing developer-oriented \ac{SSI} solutions and defines a new approach to evaluate them. For this purpose, an evaluation framework based on expert interviews, practical experience and the \ac{vc} lifecycle was developed, which enables an objective and structured evaluation of such solutions. At the same time, this work closes a gap in the existing literature. With a few exceptions, the literature has so far only focused on fundamental research and less on practical considerations of existing solutions.

For the described artefacts, further research and an expert survey were used to create an overview of various solutions on the market and their capabilities to cover the \ac{vc} lifecycle. A total of seven experts from the \ac{SSI} space who work on standards, open-source libraries and commercial solutions took part in a questionnaire and generated input for chapters \ref{chapter: expert} and \ref{chapter: framework}.
The expert input helped improve the research on 15 solutions across four categories (RQ1):
\begin{itemize}
    \item Platforms: Mattr, Trinsic, Azure AD for \acp{vc}, Verity, Affinidi
    \item SDKs: Dock.io, aca.py, Jolocom
    \item Frameworks: Veramo
    \item Libraries:  DIDKit, TangleID, Identity.com, vc.js, vc-js, verifiable-credentials-java
\end{itemize}
Among them, the solutions Mattr and Trinsic received most of the recommendations with 3 out of 7 votes in the questionnaire (RQ2).

In addition, a new developer-oriented evaluation framework based on expert opinions and practical experience was developed. For this purpose, experts were asked about important selection criteria for \ac{SSI} solutions. Moreover, a reference implementation integrating four of the solutions was developed and described. This resulted in the five categories functionality, flexibility, operability, dependency, and involvement. These in turn contain a total of 15 individual criteria, corresponding questions and a scoring scheme for a practical evaluation. For the implemented solutions Mattr, Trinsic, Veramo and Azure, Veramo received the highest score with 76.75\% and Azure the lowest with 43.78\% without weighing the individual indexes. (RQ3)

In summary, this work is the first to describe a developer-oriented examination, implementation, and evaluation of solutions in the \ac{SSI} domain. With some solutions, the concepts and technologies of \ac{SSI} can already be integrated in a production-ready manner, but the relatively young field and consequently the partially unfinished standards are still a hindrance for many solutions. In addition, it should be observed that platform solutions do not create centralized data silos and unnecessary dependencies. This would again create similar issues and situations that \ac{SSI} was meant to solve.



