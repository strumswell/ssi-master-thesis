\chapter{Conclusion}

%Digital identities und ihre Umsetzungen sind ein Jahrzente altes Problem, was bisher nur unzureichend gelöst werden konnte. Aktuelle Ansätze führen zu noch nie da gewesenen Leaks persönlicher Daten und dem Verlust von Kontrolle über von Millionen von Menschen weltweit.

Self-sovereign Identity is a new approach towards digital identities. The goal of this work was to offer developers practical tools for selecting and integrating \ac{SSI} into their products. For this, the focus was predominantly on covering the \ac{vc} lifecycle, since these very \acp{vc} contain the actual identity data and are therefore at the center of various interactions. At the same time, this work closes a gap in the existing literature. With a few exceptions, the literature has so far only focused on fundamental research and less on practical considerations of existing solutions.

In order to fulfil these goals, further research and an expert survey were used to create an overview of various solutions on the market and their capabilities to cover the \ac{vc} lifecycle. A total of seven experts from the \ac{SSI} space who work on standards, open-source libraries and their own solutions took part in the expert survey and generated input for various parts of this work. The resulting overview includes 15 solutions, including, for example, platforms such as Mattr and Trinsic, SDKs such as Jolocom and aca-py, frameworks such as Veramo, and basic libraries such as vc.js and verifiable-credentials-java (RQ1). Among them, solutions such as Mattr, Trinsic and Veramo received the most recommendations (RQ2).

In addition, a new developer-oriented evaluation framework based on expert opinions and practical experience was developed. For this purpose, experts were asked about important selection criteria for \ac{SSI} solutions and a reference implementation integrating four of the solutions was developed and described. The aim was firstly to obtain expert knowledge and generate practical experience for the development of criteria for an evaluation framework. This resulted in the five basic indexes functionality, flexibility, operability, dependency, and involvement. These in turn contain a total of 40 individual criteria, corresponding questions and a scoring scheme for a practical evaluation. For the implemented solutions Mattr, Trinsic, Veramo and Azure, Mattr received the highest score with 70.6\% and Azure the lowest with 39.98\% without weighing the individual indexes. (RQ3)

In summary, this work is presumably the first to describe a developer-oriented examination, implementation, and evaluation of solutions in the \ac{SSI} domain. With some solutions, the concepts and technologies of \ac{SSI} can already be integrated into products in a production-ready manner, but the relatively young field and consequently the partially unfinished standards are still a hindrance for many solutions. In addition, it should be observed that platform solutions do not create centralized data silos and unnecessary dependencies again. This would again create similar issues that \ac{SSI} was trying to resolve.



