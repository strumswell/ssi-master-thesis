\chapter*{Abstract}

In an increasingly globalized and digitized world, the functioning of societies often depends on digital identities. These are commonly based on centralized or federated technologies controlled by private organizations. Self-sovereign identity offers a new approach to digital identities, giving complete control back to users and breaking dependencies on often insecure middlemen. Previous literature on this has mostly focused on foundational work and less on practical developer-oriented topics. This thesis tries to close this gap by providing an overview of existing solutions and by presenting a new evaluation framework for such solutions based on a reference implementation. The goal is to provide a baseline to support developers in integrating Self-sovereign identity into their projects.

\textit{In einer zunehmend globalisierten und digitalisierten Welt ist das Funktionieren von Gesellschaften häufig von digitalen Identitäten abhängig. Diese basieren häufig auf zentralisierten oder föderierten Technologien kontrolliert durch privaten Organisationen. Self-sovereign Identity bietet dabei einen neuen Ansatz für digitale Identitäten, in dem den Nutzern die komplette Kontroller zurückgegeben wird und die Abhängigkeiten zu oft unsicheren Mittelsmännern aufgelöst wird. Bisherige Literatur hatte sich dazu vorwiegend meist auf Grundlagenarbeit und weniger auf praxisrelevante entwicklerorientierte Themen konzentriert. Diese Lücke wird in dieser Arbeit versucht zu schließen, in dem zum einen eine Übersicht über bestehende Lösungen angefertigt wird und zum anderen ein neues Bewertungsrahmenwerk für solche Lösungen auf Basis einer Referenzimplementierung vorgestellt wird. So sollen erste Grundlagen geschaffen werden, um Entwickler bei der Integration von Self-sovereign Identity in ihre Projekte zu unterstützen.}

